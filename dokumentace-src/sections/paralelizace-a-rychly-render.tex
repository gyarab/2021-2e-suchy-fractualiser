\section{Paralelizace a rychlý render}
Jeden z hlavních problémů v tomto projektu byla rychlost výpočtu zadané rovnice
pro každý pixel. Naštěstí výpočet každého pixelu je možné definovat jako čistou
funkci (anglicky pure fun\-ction)\footnote{Čisté funkce jsou funkce, které jen a
pouze vrací výsledek. Nemohou tedy např. vypsat hodnotu, nebo změnit jejich
vstup.}. Díky tomu víme, že je možné tento problém paralelizovat a jednotlivé
pixely lze bez omezení počítat současně, jelikož jeden na druhém nezávisí.

Při paralelizaci je standardně problém s orchestrací jednotlivých podprocesů,
tzn. dávání práce a získání výsledků. Naštěstí je tento problém velmi častý a
je vyřešen i pro tento specifický případ. Krom speciálního hardwaru, GPU
(Graphics Processing Unit), také existují API, které nám ho zprostředkují.

Jedno z moderních API je OpenGL. Jedná se o knihovnu, která zpřístupní
funkcionalitu GPU pomocí několika funkcí. Jazyky C++ a C sdílejí tuto samou
knihovnu, a tím pádem její používání vypadá velmi podobně v obou jazycích.
Bohužel to znamená, že nevyužívá všechny možnosti moderního C++ a OOP.

OpenGL podporuje vykreslování objektů ve 3D, což v našem případě není potřeba,
jelikož vykreslujeme jen jednu rovinu. Vykreslujeme tedy jen jeden objekt,
který pokrývá celé okno.

\subsection{OpenGL pipeline, shadery}
OpenGL render pipeline jsou kroky, co OpenGL provádí při každém renderu snímku
na obrazovce. Je definován pěti částmi, které běží na
GPU:\autocite{openglpipel}
\begin{enumerate}
    \item{Procesování vrcholů - upravuje jednotlivé vrcholy, aby dopovídaly
        např. pohledu kamery ve světě.}
    \item{Post-procesování vrcholů - další úpravy na jednotlivých vrcholech.}
    \item{Skládání vrcholů do ploch a objektů.}
    \item{Rasterizace - velmi důležitý krok u fractualiséru, jelikož se v tomto
        kroku generují výsledné pixely.}
    \item{Per-sample procesování - krok ve kterém se například míchají textury.
        Post-pro\-ce\-so\-vá\-ní jednotlivých pixelů.}
\end{enumerate}

Každý krok lze ovlivnit pomocí shaderů, což jsou programy, které běží přímo na
GPU. Při procesování vrcholů se spouští vertex shader. V našem případě je velmi
jednoduchý. Vykreslujeme jen jednu plochu a vrcholy, co zadáme k vykreslení
nemusíme nijak měnit.

\begin{figure}[h!]
\centering
\includegraphics[width=0.45\textwidth]{vrstvy.png}
\caption{Vrstvy, které vznikají při iteraci zadané rovnice}
\label{fig:vrstvy}
\end{figure}

Další z shaderů je fragment shader, který běží ve čtvrtém kroku renderovací
pipeline - rasterizaci. Toto je nejzásadnější část kódu, protože právě ta nám
umožní každý pixel na obrazovce zabarvit podle vypočítané rovnice. Tento shader
musí pro naše účely udělat několik věcí: 1. převést pozici pixelu, který počítá
na pozici v grafu, 2. spočítat, jestli bod patří do množiny a kolik iterací
bylo potřeba, abychom odhalili, že v množině není, 3. vybrat správnou barvu
výsledného pixelu z textury (tento krok je standardní ve všech fragment
shaderech). V kroku 2 bychom mohli rozhodovat čistě jen hodnotou ano či ne,
ale zapamatování si kolik iterací bylo potřeba je mnohem zajímavější a na obrázku
můžeme poté vidět jednotlivé vrstvy, jak se fraktál odhaluje (obrázek \ref{fig:vrstvy}).

\subsection{Generování shaderů}
Další z problémů, který bylo nutné vyřešit, bylo generování shader kódu ze
zadané matematické funkce. Vzhledem k tomu, že se shader musí kompilovat při
běhu programu, nebyl velký problém samotná kompilace, ale získání správného
kódu. Zvolil jsem cestu rekurzivního přepsání originální funkce na zdrojový
kód, který je poté vložen do zbytku shaderu. Zbytek shaderu je přečten ze
souboru v pracovním adresáři pod názvem \texttt{shader.glsl}. Zkušený uživatel
si tak tímto může shader upravit a generovat i jiné fraktaly, než ty, které
v tuto chvíli fractualiser podporuje.

Princip přepisu spočívá ve standarním průchodu aritmetického výrazu, jako
u vytváření aritmetického stromu. Místo vytváření uzlů ale vytváříme kusy
kódu, které poté spojíme dohromady, abychom dostali konečný výraz přepsaný
do GLSL kódu.

Jazyk GLSL standardně nepodporuje komplexní čísla, ale dokáže pracovat s
vektory. Kaž\-dé komplexní číslo je tedy znázorněno jako dvouprvkový vektor.
Kvůli tomuto omezení bylo potřeba vytvořit funkce pro násobení a dělení,
jelikož násobení vektorů není ekvivalentí násobení komplexních čísel. Sčitání a
odčítání však pro dvouprvkové vektory je.

Fukce $z*z*z + 3z - 0.2 + 0.1i$ by například po přepisu do GLSL kódu vypadala
následovně:
\texttt{mult(mult(z, z), z) + mult(dvec2(3, 0), \\z) - dvec2(0.1, 0) + dvec2(0, 0.2)}.

Fractualiser zatím podporuje jen násobení, dělení, sčítání a odčítání, ale
přidání dalších funkcí by nebyl problém. Zároveň jsou tyto operace dostatečné
na vykreslení mnoha funkcí. Jedna z optimalizací, která se nabízí, je předpočítání
konstant. Ve vygenerovaném kódu v minulém odstavci můžeme vidět, že $-0.2$ a $0.1i$
byly přepsány jako dva odlišné vektory. Přitom mohou být spojeny do jednoho:
\texttt{dvec2(-0.2, 0.1)}.
