\section{Paralelizace a rychlý render}
Jeden z hlavních problémů v tomto projektu byla rychlost výpočtu zadané rovnice
pro každý pixel. Naštěstí výpočet každého pixelu je možné definovat jako čistou
funkci (anglicky pure fun\-ction)\footnote{Čisté funkce jsou funkce, které jen a
pouze vrací výsledek. Nemohou tedy např. vypsat hodnotu, nebo změnit jejich
vstup.}. Díky tomu víme, že je možné tento problém paralelizovat a jednotlivé
pixely lze bez omezení počítat současně, jelikož jeden na druhém nezávisí.

Při paralelizaci je standardně problém s orchestrací jednotlivých podprocesů,
tzn. dávání práce a získání výsledků. Naštěstí je tento problém velmi častý a
je vyřešen i pro tento specifický případ. Krom speciálního hardwaru, GPU
(Graphics Processing Unit), také existují API, které nám ho zprostředkují.

Jedno z moderních API je OpenGL. Jedná se o knihovnu, která zpřístupní
funkcionalitu GPU pomocí několika funkcí. Jazyky C++ a C sdílejí tuto samou
knihovnu, a tím pádem její používání vypadá velmi podobně v obou jazycích.
Bohužel to znamená, že nevyužívá všechny možnosti moderního C++ a OOP.

OpenGL podporuje vykreslování objektů ve 3D, což v našem případě není potřeba,
jelikož vykreslujeme jen jednu rovinu. Vykreslujeme tedy jen jeden objekt,
který pokrývá celé okno.

% TODO: shadery, render pipeline
