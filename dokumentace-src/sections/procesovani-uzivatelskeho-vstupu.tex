\section{Procesování vstupu}
Fractualiser lze ovládat klávesnicí i myší. Vstup z klávesnice i myši je
nejdříve zachycen knihovnou GLFW, která nám poskytuje abstrakci od jednotlivých
API operačních systémů, nebo desktop managerů. Místo psaní toho samého kódu
několika způsoby pro Windows, X11 pro Linux, nebo pro MacOS můžeme použít GLFW
a ta nám připraví okno, ve kterém můžeme pracovat, a funkce, pomocí kterých
můžeme poslouchat příchozí vstupy od uživatele.

S myší lze obrazem pohybovat, nebo ho přiblížit či oddálit kolečkem. To samé
lze učinit pomocí kláves WASD a QE. Pomocí klávesnice můžeme vyvolat další
akce:
\begin{itemize}
    \item{\textbf{F/C} - ovládání počtu iterací. Klávesa F zvyšuje 
        maximální počet iterací pro každý pixel. Klávesa C ho snižuje.}
    \item{\textbf{G} - zobrazí informace o stavu programu}
    \item{\textbf{P} - začne export obrázku}
    \item{\textbf{ESC} - ukončí program}
\end{itemize}

GLFW nám umožňuje číst vstup vícekrát za jeden snímek. Toto je užitečné a
dovoluje nám mít rezponzivnější program. Zároveň je ale potřeba si zapamatovat
všechny vstupy do té doby, co program nevykreslí další snímek. Mimo hlavní
nastavení zobrazovaného grafu je také dočasné, které akumuluje změnu, kterou je
potřeba provést před dalším snímkem. Po každém vykreslení snímku se toto
nastavení vynuluje a následně se do něj načtou další vstupy.

\vfill\eject
