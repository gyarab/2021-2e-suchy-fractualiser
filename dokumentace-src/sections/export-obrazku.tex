\section{Export obrázku}
Další cíl fractualiseru byla možnost vyexportovat obrázek, který může mít
několikrát větší rozlišení, než monitor uživatele. Vyexportovaný obrázek
lze poté využít například pro tisk, nebo do jiných kreativních projektů.

OpenGL vykresluje obraz do \texttt{FrameBuffer}u. Tento buffer je poté
zobrazen v okně aplikace. Abychom se vyhli limitace velikosti okna, tak
místo vykreslování do výchozího bufferu vykreslíme fraktál do textury,
kterou poté můžeme po částech číst a zapsat do souboru.

Zvolený formát exportovaného obrázku je Windows Bitmap kvůli jeho
jednoduchosti. Z tohoto samého důvodu výsledný obrázek není komprimovaný a tím
pádem může být i řádově větší než ten samý obrázek konvertovaný do formátu PNG.
Serializace do Windows Bitmap formátu byla implementována podle specifikace \\
\textcite{bitmapstorage}. Samotný formát je velmi minimalistický a jsou potřeba
jen dvě hlavičky.

\begin{lstlisting}
struct Header {
    char id[0x02] = {0x42, 0x4D};
    int file_size;
    char more_identifiers[4];
    int bitmap_start;
};
struct DIBHeader {
    int size_of_header;
    int width;
    int height;
    unsigned short num_clr_panes;
    unsigned short pixel_bitlen;
    int compression;
    int bitmap_size;
    int v_res;
    int h_res;
    int n_colors;
    int imp_colors;
};
\end{lstlisting}

\texttt{Header} slouží k identifikaci \texttt{.bmp} formátu a poté k základním
metadatům o jeho velikosti. \texttt{DIBHeader} popisuje danou bitmapu a
specifikuje detaily o jejím rozlišení, nebo barvách.

Jeden ze zvlástních problémů, který nastal při implementaci byl problém s
výchozím zarovnáním \texttt{struct}ů na 4 byty. Toto mělo ale lehké řešení a to
přidání \texttt{\#pragma pack(1)}, a\-by\-chom compiler upozornili, že nechceme, aby
tento \texttt{struct} zarovnával.

Alternativa k \texttt{.bmp} formátu by mohl být formát QOI (Quite OK Image formát),
který je velmi jednoduchý, ale zároveň dokáže obsah komprimovat skoro stejně dobře,
jako mnohem složitější PNG formát. Jediná jeho nevýhoda je špatná kompatibilita se
znamými programy.
